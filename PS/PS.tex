\documentclass[a4paper,12pt]{article}
\usepackage{amsmath}
\usepackage{amsfonts}
\usepackage{array}
\usepackage{ulem}
\usepackage{listings}
\usepackage{geometry}
\usepackage{chngpage}
\ifx\pdfoutput\undefined
\usepackage[dvips]{graphicx}
\DeclareGraphicsExtensions{.eps,.ps,.bmp,.tif,.tiff,.tag}
\else
\usepackage[pdftex]{graphicx}
\usepackage[colorlinks,urlcolor=blue,linkcolor=blue,anchorcolor=black,citecolor=black,filecolor=blue,runcolor=blue]{hyperref}
\DeclareGraphicsRule{*}{mps}{*}{}
\DeclareGraphicsExtensions{.jpg,.pdf,.mps,.png}

\linespread{1.1}
\geometry {left=2cm, right=2cm , top=2cm ,bottom=3cm}
\begin{document}

\centerline{\large{Statement of Purpose for Department of CS, Columbia University}}
\vspace{2ex}
\centerline{\large{Tianyu Wang(wtyatzoo@zju.edu.cn)}}
\vspace{2ex}

I want to pursue a Ph.D. degree in computer science, and my career aspiration is to become a research scientist in graphics community finally.\vspace{1.2ex}

My interest mainly is physical based forward simulation of solid or fluid and inverse optimization problems for physical or geometric design, for example inverse optimization of rest shape or material distribution or something else to lead that the final simulation results under the specific boundary condition can be achieved. Of course, how to model these problems is the most exciting point and then a better practical numerical method.\vspace{1.2ex}

It is fantastic to exhibit the real world in computational environment and it is also the primal goal for computer graphics. Not only this which is mainly for special effect or games, up to now, graphics community has extended to more broader field like computational design, computational photography and many other fields' technologies also influence this community's development like machine learning for graphics. So today computer graphics' research has a more extensive domain of definition and I believe there must still exist many troublesome challenges need to be solved in industry. I have pleasure to spend more years to pursue the solutions for the challenging problems on the cutting edge of graphics to help artists, engineers and common users to get better user experience and controllability. \vspace{1.2ex} 

I have three years' full-time research training advised by Prof Jin Huang in Zhejiang University and published one paper in the ACM Transactions on Graphics 2018 as the third author (the second student author). During these three years, I have some experience in solid and fluid simulation, geometry processing and deep learning.\vspace{1.2ex} 

% the deep exploration and team work and almost bug-free experiment by myself  
In the work that is exhibited at SIGGRAPH 2018, we extended the original 2009 SIGGRAPH's numerical coarsening method for FEM based simulation in linear manner into applying to non-linear constitutive model like neo-hookean and so on. I closely worked with my senior classmate Jiong Chen. In this process, I implemented a basic FEM based solid simulation framework quickly first and then did the major two papers' comparison experiments almost by myself. And we frequently communicated to analyze the experiment results. I am confident that I have good teamwork stills and the potential to explore deeply.\vspace{1.2ex}

% the quick reading and coding to reproduce the results of the papers   
I have skills in quick reading and coding to reproduce the results of the graphics related papers. For example, in the national key program of China project, I quickly reproduced the result of three kinds of mass-spring based simulation: using classical newton method, using fast mass-spring method based on the paper \textit{Fast Simulation of Mass-Spring Systems}, using modified fast mass-spring based on the paper \textit{A Chebyshev Semi-Iterative Approach for Accelerating Projective and Position-based Dynamics} with a CUDA version jacobi solver acceleration.\vspace{1.2ex} %details

% extensive research interest: my master thesis 
I have extensive research interest. So I try to investigate the ability of deep learning for shallow water simulation data's super resolution task in my master thesis project. This thesis proposes to achieve the SWE data super-resolution using GAN. After thinking about that the smoothness demand on time dimension, rotation equivalence requirement, possible negative value, this thesis proposes 1.introducing the time discriminator, 2.using SWE's velocity information, 3. using the asynchronous training method based on full training set's gradient, 4.using rotation symmetric data augmentation method, 5.using Leaky ReLU activation function instead ReLU, so that more high-frequency details can be inferred from the low-resolution down sampled data in the results.\vspace{1.2ex}  %details 

% knowledge of mathematics and physics
Besides, I also have some experience in reading geometry processing related paper and reproducing some papers' algorithm, for example that I reproduced Prof. Keenan Crane's heat flow based geodesic computation algorithm which is elegant and concise in maths and Prof. Jin Huang' L1 Polycube deforamtion based method for hexahedralization partly. Fluid simulation is fancy to me! So in my spare time, I read the major papers introducing the grid, particle or hybrid method in graphics community and did some experiment, for example basic SPH or grid based method. Generally speaking, I get some basic skills and experience in numerical optimization, numerical integration for solid or fluid simulation, geometry processing and neural network's training.\vspace{1.2ex}

I want to pursue my Ph.D. degree at Columbia because of the strong computer graphics group within the Computer Science department. Here are two talented professors, Prof. Eitan Grinspun and Prof. Changxi Zheng. They both do research in graphics community, mainly focusing on forward simulation and related inverse design problems. For more, they have their respective strengths that Eitan is talented at using the geometric view to see the real world's physical phenomenon and did significant works in modeling rod or shell's motion using discrete differential geometry and Changxi did fantastic research in acoustic simulation for many different physical phenomenon. And this group has collaboration with many top research institutions around the world. Therefore, a Ph.D. student can get a deep and wide research training here from both two professors' complementary strengths.\vspace{1.2ex} 

You can find further information of my research experience from my homepage\\(https://wtyatzoo.github.io/) if you like.\vspace{1.2ex}

Thank you for your consideration of my application.

\vspace{1.2ex}
\end{document}
