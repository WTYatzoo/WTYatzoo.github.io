\documentclass[a4paper,12pt]{article}
\usepackage{amsmath}
\usepackage{amsfonts}
\usepackage{array}
\usepackage{ulem}
\usepackage{listings}
\usepackage{geometry}
\usepackage{chngpage}
\ifx\pdfoutput\undefined
\usepackage[dvips]{graphicx}
\DeclareGraphicsExtensions{.eps,.ps,.bmp,.tif,.tiff,.tag}
\else
\usepackage[pdftex]{graphicx}
\usepackage[colorlinks,urlcolor=blue,linkcolor=blue,anchorcolor=black,citecolor=black,filecolor=blue,runcolor=blue]{hyperref}
\DeclareGraphicsRule{*}{mps}{*}{}
\DeclareGraphicsExtensions{.jpg,.pdf,.mps,.png}

\linespread{1.1}
\geometry {left=2cm, right=2cm , top=2cm ,bottom=3cm}
\begin{document}

\centerline{\large{Statement of Purpose for Department of CS, Columbia University}}
\vspace{2ex}
\centerline{\large{Tianyu Wang(wtyatzoo@zju.edu.cn)}}
\vspace{2ex}

I want to pursue a Ph.D. degree in computer science, and my career aspiration is to become a research scientist in graphics community finally.\vspace{1.2ex}

My current interest mainly is physical based forward simulation of solid or fluid and inverse optimization problems for physical or geometric design, for example inverse optimization of rest shape or material distribution or something else to lead that the final simulation results under the specific boundary condition can be achieved. Of course, how to model these problems is the most exciting point and then a better practical numerical method.\vspace{1.2ex}

Besides, we can see that current graphics community has extended to more broader fields like computational design, computational photography and many other fields' technologies also influence this community's development like machine learning for graphics. With the development of this industry, I believe there must exist many troublesome challenges need to be solved. I have a passion to spend more years to pursue the solutions for the challenging problems at the cutting edge of graphics to help artists, engineers and common users to get better user experience and controllability as a research scientist. Previously, I have accumulated some experience of graphics research as follows. \vspace{1.2ex}  

I have three years' full-time research training advised by Prof Jin Huang in Zhejiang University and published one paper in the ACM Transactions on Graphics 2018 as the third author (the second student author). During these three years, I have some experience in solid(FEM based elasticity, mass-spring model, discrete rod model, discrete shell model, subspace acceleration) and fluid (SPH, grid, hybrid method, SWE model) simulation, geometry processing (smoothing, fairing, basic parameterization, geometric energy based deformation, subdivision, geodesic distance computation) and deep learning (CNN, GAN, super-resolution, style transfer). Generally speaking, I get some basic skills and experience in numerical optimization, numerical integration for solid or fluid simulation, geometry processing and neural network's training. My major graphics related code with license to open can be found in my github page(https://github.com/WTYatzoo/). \vspace{1.2ex} 

% the deep exploration and team work and almost bug-free experiment by myself  
In the work that is exhibited at SIGGRAPH 2018, we proposed the new numerical coarsening method for FEM based simulation with non-linear constitutive model. In this process, I implemented a basic FEM based solid simulation framework quickly first and then did the major two papers' comparison experiments almost by myself. I closely worked with the first author, my senior classmate Jiong Chen. We frequently communicated to analyze the experiment results and to organize the needed illustrations. I think that I have good teamwork stills and the potential to explore challenging problems deeply.\vspace{1.2ex}

% the quick reading and coding to reproduce the results of the papers   
I have strong skills in reading and coding to reproduce the results of the graphics related papers. For example, in the national key program of China project, I quickly reproduced the results of three kinds of mass-spring based simulation: using classical Newton method, using local-global strategy based on the paper \textit{Fast Simulation of Mass-Spring Systems}, using modified fast mass-spring based on the paper \textit{A Chebyshev Semi-Iterative Approach for Accelerating Projective and Position-based Dynamics} with a CUDA version jacobi solver acceleration. I also partly guided my junior classmates to continue this project.\vspace{1.2ex} %details

% unfamiliar research topic: my master thesis 
I have ability to quickly understand and dig into an unfamiliar research topic and make some incremental advances compared to the state of the art. In my master thesis project, I try to investigate the ability of deep learning for shallow water simulation data's super resolution task. This thesis proposes to achieve the SWE data super-resolution using GAN. By taking account of the temporal smoothness demand, rotation equivalence requirement, the possible negative value, this thesis proposes 1.introducing the time discriminator, 2.using SWE's velocity information, 3. using the asynchronous training method based on full training set's gradient, 4.using rotation symmetric data augmentation method, 5.using Leaky ReLU activation function instead ReLU, so that more high-frequency details can be inferred from the low-resolution downsampled data in the results. See the \textit{\href{https://wtyatzoo.github.io/reports/SWE.pdf}{report}} or \textit{\href{https://wtyatzoo.github.io/thesis/master\_thesis.pdf}{my master thesis}} for details.\vspace{1.2ex}  %details 

% knowledge of mathematics and physics
Besides, I have extensive research interest. I also have some experience in reading and implementing geometry processing related papers. For example, I reproduced Prof. Keenan Crane's heat flow based geodesic distance computation algorithm which is elegant and concise in maths and the core part of Prof. Jin Huang' L1 Polycube deforamtion based method for hexahedralization which is mainly a L1 norm optimization problem. Besides, fluid simulation is fascinating to me! So in my spare time, I read and implemented the representative papers introducing the grid, particle or hybrid methods in graphics community. \vspace{1.2ex}

I want to pursue my Ph.D. degree at Columbia because of the strong computer graphics group within the Computer Science department. Here are two talented professors, Prof. Eitan Grinspun and Prof. Changxi Zheng. They both do research in graphics community, mainly focusing on forward simulation and related inverse design problems. For more, they have their respective strengths that Eitan is talented at using the geometric view to see the real world's physical phenomena and did significant works in modeling rod or shell's motion using discrete differential geometry and Changxi did broad works on forward simulation, inverse design, networking and so on, especially the fantastic acoustic simulation for many different physical phenomena. In addition to this, this group has collaboration with many top research institutions around the world. Therefore, a Ph.D. student can get a deep and wide research training here from both two professors' complementary strengths.\vspace{1.2ex} 

You can find further information of my research experience from my homepage\\(https://wtyatzoo.github.io/) if you like.\vspace{1.2ex}

Thank you for your consideration of my application.

\vspace{1.2ex}
\end{document}
